\documentclass[12pt,a4]{article}
\usepackage[left=1.8cm,right=1.8cm,top=32mm,columnsep=20pt]{geometry}

\usepackage[utf8]{inputenc} %Formato de codificación
\usepackage[spanish, es-tabla, es-nodecimaldot]{babel}
\usepackage{amsmath} %paquete para escribir ecuaciones matemáticas
\usepackage{float} %Para posicionar figuras
\usepackage{graphicx} %Para poder poner figuras
\usepackage{tikz}
\usetikzlibrary{positioning}
\usetikzlibrary{shapes.geometric, decorations.pathreplacing}

\title{Encontrando el coeficiente de fricción dinámica}
\author{Francisco Carruthers, Facundo Firpo y Joel Jablonski\\ [2mm]
\small \texttt{\{fcarruthers, ffirpo, jjablonski\}@udesa.edu.ar}\\
\small Fisica I, tutorial Vinograd}
\date{2do Semestre 2024}


\begin{document}

\maketitle

\begin{abstract}
    Se investigó el coeficiente de fricción dinámica entre un trineo y varias superficies utilizando un sistema de trineo, soga y polea. En primer lugar, se realizó una calibración previa del sistema, ajustando los sensores y asegurando que las lecturas de posición y tiempo fueran precisas. A continuación, se varió la masa del trineo y se midió la aceleración para calcular el coeficiente de fricción dinámica ($\mu_d$) en diferentes superficies. Las superficies evaluadas incluyeron el trineo sobre madera, trineo sobre papel y papel sobre papel. Los valores obtenidos para el $\mu_d$ fueron: 0.4 ± 0.1 para madera y trineo, 0.45 ± 0.03 para papel y trineo, y 0.5 ± 0.2 para papel y papel. A medida que incluimos mas papel en la friccion, el $\mu_d$ aumenta. Estos resultados sugieren que la textura del papel es mas rugosa que la madera pulida de la mesa.
\end{abstract}

\section{Introducción}

La fricción es una fuerza de resistencia que actúa en oposición al movimiento relativo entre dos superficies en contacto. Existen dos tipos principales de fricción: la fricción estática, que previene el movimiento, y la fricción dinámica (o cinética), que actúa cuando el objeto ya está en movimiento. En este experimento, nos centraremos en la fricción dinámica, cuyo coeficiente, denotado como \(\mu_d\), se define como la relación entre la fuerza de fricción y la fuerza normal ejercida sobre el objeto:

\[
\vec{F_r} = \mu_d \cdot \vec{F_n}
\]

donde \(\vec{F_r}\) es la fuerza de fricción y \(\vec{F_n}\) es la fuerza normal, que para superficies horizontales es equivalente al peso del objeto en contacto con la superficie. Este coeficiente depende del tipo de materiales en contacto y su textura.

La determinación del $\mu_d$ es fundamental en la física aplicada, ya que afecta el diseño de sistemas mecánicos, el análisis de movimientos y la estabilidad de objetos en distintas superficies. En este contexto, comprender la relación entre la fuerza de fricción y la aceleración del objeto es crucial. Para este experimento, utilizamos la segunda ley de Newton, que establece que la fuerza neta que actúa sobre un objeto es proporcional a la masa del objeto y su aceleración:

\[
\vec{F} = m \cdot \vec{a}
\]

\section{Practica experimental}

\begin{figure}[H]
    \centering
    \includegraphics[width=0.9\linewidth]{esquema.png}
    \caption{Esquema experimental para la propuesta. La masa m y M son variables.}
    \label{fig:system}
\end{figure}

Dispusimos de un sistema de trineo, soga, polea y sensor de posicion. El sistema se muestra en la Figura \ref{fig:system}. La masa del trineo y la masa total del sistema son variables, lo que nos permite explorar varias configuraciones para ver como las masas afectan al $\mu_d$.

Para adquirir datos usamos el microcontrolador Arduino Mega, al que le conectamos un
sensor de posición de ultrasonido. Este aparato envía ondas sonoras desde el transmisor, que luego rebotan
en un objeto y regresan al receptor. Se puede determinar qué tan lejos está algo por el tiempo que tardan las
ondas sonoras en regresar al sensor.

En primera instancia, nos propusimos calibrar el sensor. La señal es proporcional a la distancia entre el sensor y el trineo. La calibración consistió en determinar la pendiente y la ordenada al origen de la relación entre la señal y la distancia. Estos valores son necesarios para determinar la distancia recorrida por el trineo en función de la señal del sensor. Para ello, con una regla fuimos ubicando el trineo a distancias conocidas y anotamos la señal del sensor. Luego, ajustamos una recta a los datos obtenidos.

Una vez calibrado el sensor, procedimos a realizar el experimento. Primero, colocamos el trineo sobre la mesa de madera y lo dejamos deslizar. Medimos la distancia recorrida por el trineo en función del tiempo. Este procedimiento lo repetimos para $m = 161 \pm 1 g$ y $M = 72 \pm 1 g$, $m = 243 \pm 1 g$ y $M = 95 \pm 1 g$ y $m = 109 \pm 1 g$ y $M = 46 \pm 1 g$. Con estos datos, haciendo un ajuste cuadratico, podemos determinar la aceleración del trineo. 
Luego, repetimos el experimento pero pegandole un papel a la mesa. Usamos $m = 161 \pm 1 g$ y $M = 72 \pm 1 g$, $m = 243 \pm 1 g$ y $M = 95 \pm 1 g$ y $m = 109 \pm 1 g$ y $M = 46 \pm 1 g$. 
Por ultimo, pegamos un papel al trineo y repetimos el experimento.

Con los datos obtenidos, podemos calcular el $\mu_d$ para cada combinación de masas y superficies. Para ello, utilizamos la ecuación \ref{eq:mu_d}, donde $\vec{a}$ es la aceleración del trineo y $\vec{g}$ es la aceleración de la gravedad.

\begin{equation}
    \mu_d = \frac{\vec{a} \cdot (M + m) + M \cdot \vec{g} }{m \cdot \vec{g}}
    \label{eq:mu_d}
\end{equation}

La aceleración de la gravedad esta dada por $9.8 m/s^2$ y la aceleración del trineo la obtenemos realizando un ajuste cuadratico a los datos de posicion obtenidos. Al realizar un ajuste cuadratico obtenemos una ecuacion del tipo $pos(t) = a \cdot t^2 + b \cdot t + c$. La aceleración es el doble del coeficiente de $t^2$.

\section{Calibración}

Para calibrar el sistema, empleamos un sistema de referencia basado en mediciones de distancia que varian entre 15 cm y 35 cm, utilizando una regla previamente calibrada como
 instrumento de medición. El sensor en cuestión proporciona unidades arbitrarias. Al comparar estos valores con las distancias reales, determinamos que existe una relación lineal
  entre la señal del sensor y la distancia medida. Con esta información, realizamos un ajuste lineal a los datos obtenidos, lo que nos permitió derivar una ecuación de la recta que
   describe esta relación. Los valores resultantes de este ajuste, junto con sus respectivas incertidumbres, se presentan a continuación:

\begin{figure}[H]
    \centering
    \includegraphics[width=0.8\linewidth]{Calibracion.png}
    \caption{Calibracion del sistema mediante la comparacion de la señal del sensor con la distancia real}
    \label{fig:calibracion}
\end{figure}

El grafico de la Figura \ref{fig:calibracion} muestra la recta:

\begin{equation}
    dist(a) = (0.0184 \pm 0.0005) \cdot a + (-0.5 \pm 0.5) cm
    \label{eq:dist}
\end{equation}

La pendiente representa la tasa de cambio de la señal del sensor con respecto a la distancia medida, indicando que por cada centímetro de distancia, la señal del sensor cambia en
 aproximadamente 0.0184 unidades arbitrarias. La incertidumbre asociada a la pendiente, de ±0.0005 unidades arbitrarias, refleja la precisión del ajuste lineal realizado.


Por otro lado, la ordenada al origen de -0.5 cm, con una incertidumbre de ±0.5 cm, sugiere que cuando la señal del sensor es cero, la distancia medida sería aproximadamente -0.5 cm.
 Esta ordenada al origen negativa puede interpretarse como un desplazamiento sistemático en las mediciones del sensor, posiblemente debido a factores como el posicionamiento inicial 
 del sensor o pequeñas desviaciones en la calibración de la regla utilizada.

\section{Resultados}

\subsection{Posicion}

Registramos la distancia recorrida y el tiempo empleado para calcular la velocidad y la aceleración del trineo. Al tener estas mediciones, podemos evaluar cómo factores como la
 fricción influyen en el rendimiento del trineo en las diferentes superficies.

Posteriormente, realizamos un ajuste cuadrático a los datos obtenidos.
 Al incluir un término cuadrático en el modelo, pudimos capturar posibles curvaturas en los datos que un ajuste lineal no podría representar adecuadamente. Este enfoque nos
  proporcionó una visión más precisa del movimiento del trineo.

\subsubsection*{Consecuencia del cambio de masas}

Al modificar la masa del trineo \textit{m} en nuestro experimento, observamos cambios significativos en la aceleración. En nuestros datos, al incrementar \textit{m}, las curvas de aceleración tienen una menor inclinacion, indicando una menor aceleración del trineo.

\begin{figure}[H]
    \centering
    \includegraphics{ajuste2_PisoMaderaMPB_O.png}
    \caption{Grafico de posicion del trineo deslizando sobre la mesa con \textit{M} = 72 y variando \textit{m}}
    \label{fig:M_OP piso trineo}
\end{figure}

Con esto podemos afirmar que la masa del trineo influye en la aceleración del sistema. A mayor masa, menor aceleración, debido a que es mas dificil de traccionar con la misma fuerza provocada por el peso de \textit{M}. 

Como el hilo que usamos para tirar tiene masa despreciable y esta en tension todo el tiempo por lo que su longitud no varia signficativamente, podemos afirmar que lo contrario ocurre si aumentamos la masa de \textit{M}. A mayor masa de \textit{M}, mayor aceleración del sistema.

Del ajuste cuadratico visto en la figura \ref{fig:M_OP piso trineo} obtenemos las aceleraciones $(11 \pm 1) \frac{cm}{s^2}$ para $\textit{m} = (243 \pm 1) g$ y $(93 \pm 5) \frac{cm}{s^2}$ para $\textit{m} = (161 \pm 1) g$.

\subsubsection*{Consecuencias del cambio de superficie}

Para evaluar cómo diferentes superficies afectan el rendimiento del trineo, realizamos experimentos cubriendo la mesa con papel y comparando los resultados con los obtenidos sobre una superficie de madera. En la Figura \ref{fig:M_OP piso hoja}, se presentan las curvas de aceleración para ambas superficies con los mismos \textit{m} y \textit{M}. Estos resultados demuestran que la elección de la superficie tiene un impacto directo en la dinámica del trineo, afectando su velocidad y eficiencia de movimiento en el contexto de nuestro experimento.

\begin{figure}[H]
    \centering
    \includegraphics{ajuste2_PisoHojaM_OP.png}
    \caption{$m = (161 \pm 1) g, M = (72 \pm 1) g$}
    \label{fig:M_OP piso hoja}
\end{figure}

Del ajuste cuadratico visto en la figura \ref{fig:M_OP piso hoja} obtenemos las aceleraciones $(26 \pm 2) \frac{cm}{s^2}$ para papel y $(46 \pm 5) \frac{cm}{s^2}$ para madera.

Observamos que la aceleracion del papel es menor que la de la madera. Viendo la ecuacion \ref{eq:mu_d}, para los mismos \textit{m} y \textit{M}, la aceleracion marca la diferencia del $\mu_d$, por lo que la friccion del papel es mayor que la de la madera de la mesa.

\subsection{Obtencion del $\mu_d$}

Sacando un promedio de los valores obtenidos de la ecuacion \ref{eq:mu_d} usando cada aceleración, obtenemos un valor de $\mu_d$ para cada superficie. Este procedimiento es fundamental para determinar cómo varía el coeficiente de fricción dinámico $\mu_d$ en función de las diferentes combinaciones de materiales. A continuación, se presentan los resultados obtenidos:
\begin{figure}[H]
    \begin{table}[H]
        \centering
        \begin{tabular}{|c|c|}
            \hline
            \textbf{Superficie} & \textbf{$\mu_d$}\\
            \hline
            Madera y trineo & $0.3 \pm 0.08$\\
            Papel y trineo & $0.42 \pm 0.02$ \\
            Papel y Papel & $0.5 \pm 0.1$ \\
            \hline
        \end{tabular}
        \caption{Valores de $\mu_d$ y sus incertezas para cada superficie}
        \label{tab:mu_d}
    \end{table}
\end{figure}

En la tabla \ref{tab:mu_d}, se puede observar que el coeficiente de fricción dinámico varía según la superficie en contacto. Por ejemplo, la combinación de madera y trineo presenta un $\mu_d$ de 0.4 con una incerteza de ± 0.1, lo que indica una fricción moderada. En contraste, la combinación de papel y papel muestra un $\mu_d$ de 0.5 con una incerteza de ± 0.2, sugiriendo una fricción más alta pero también una mayor variabilidad en los resultados.

Estos valores son cruciales para entender cómo diferentes materiales interactúan entre sí en términos de fuerza de fricción. Se destaca el hecho de que el mayor $\mu_d$ se observó en la combinación de papel y papel. La rugosidad superficial del papel es un factor clave. El papel está diseñado para absorber tinta, lo que implica una textura más rugosa y porosa en comparación con superficies más lisas como la madera o el trineo. Esta rugosidad aumenta el área de contacto efectiva entre las dos superficies de papel, incrementando así la fricción.
\section{Conclusiones}

A través de las mediciones de aceleración en diversas configuraciones de masa y superficie, se obtuvieron valores para el $\mu_d$, destacándose variaciones entre las diferentes superficies analizadas.

Los resultados obtenidos evidencian que la fricción varía dependiendo de la textura de las superficies en contacto. Por ejemplo, el valor de $\mu_d$ en la superficie de madera fue de 0.4 ± 0.1, mientras que en papel sobre papel fue mayor, alcanzando un promedio de 0.5 ± 0.2. Esta variación en el coeficiente es indicativa de la influencia de la rugosidad y la naturaleza del material en la interacción de fricción.

Por las incertezas de los resultados, podemos comprender la importancia de las condiciones experimentales y cómo pequeñas variaciones pueden impactar en los resultados, reforzando el valor de una correcta medición y calibración en estudios físicos.


\end{document}